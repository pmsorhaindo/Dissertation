\documentclass[main.tex]{subfiles} 
\begin{document}
%% The Intro Content
%%Acknowledgements TODO

%------The project and what it should entail.
The final year project is described as an extensive piece of individual work relevant to the course. One of liberties this project offers is a chance to take a particular area of interest and significantly deepen the knowledge in the subject. Therefore, an important decision that is made early on in any final year project, is the choice of topic and subject areas on which to focus. An interest in the subject area can play a key part in a projects success or failure and for this reason the process of settling on a topic involved a significant amount of time and research.

%------Alternative choices why I didn't pick them
\paragraph{} The initial idea which was developed into a project proposal involved the production of software within the Natural Language Processing field. The primary aim was to implement an intelligent agent that would parse text, recognizing relationships between pronouns and the nouns they refer to. The program would store everything in a database that it could then parse in order to analyse the collected data statistically. Based on user feedback, as to whether the system was parsing correctly or not, the program would then self optimize the initial parsing rules given to it in order to parse more efficiently and accurately. Furthermore, this could be extended further to include text generation functionality allowing it to communicate with a user via a keyboard and screen. It would effectively function as a chat-bot which ``learnt'' from its mistakes via Artificial Intelligence algorithms such as reinforced learning and Genetic Algorithms. After further research into the Natural Language Processing field it was realized just how ambitious these plans were. The subject of data mining a corpus of descriptive text in isolation is a broad field subject to much academic research.

\paragraph{}Following interest developed exploring the field of Artificial Intelligence in particular genetic algorithms a second concept took form. The aims were to recreate a race track and racing car and implement an intelligent agent with a limited view of the track such as a human would have. After exposing the functionality of the car to the agent it would then be given the goal of completing the track in the least amount of time possible. This idea was dismissed in favour of the final idea because it required more development on an environment and physics involved in simulating a car accurately than the development of an intelligent agent that it was possible that the Artificial Intelligence element would not get the time required to produce a finished artefact.

%------My choice  of Project
\paragraph{}After reading around the field of Artificial Intelligence the topic of swarm intelligence was discovered which was first used in reference to robots. The definition given of Swarm Intelligence is ``systems of non-intelligent robots exhibiting collectively intelligent behaviour evident in the ability to unpredictably produce `specific' ([i.e.] not in a statistical sense) ordered patterns of matter in the external environment''\cite{Beni1989} the idea of implementing a system that could simulate a group of simple objects that together worked intelligently was greatly appealing as a final year project. Ants and ant colonies provide an interesting application to the real world for the abstract idea behind swarm intelligence. Also the large amount of ants often found in colonies offered the opportunity to look at the problem on a large scale, providing a large amount of computation in terms of decisions, movement and collisions.

\paragraph{}During the process of drawing up the project proposal the decision was made to focus on how the problem could still be run efficiently while increasing the amount of computation being carried out. When reasoning about the problem it was clear that many decisions would need to be made at once, these circumstances would be best dealt with simultaneously, hence using parallel processing it would be possible to exploit this demand for things to be computed at the same time in order to efficiently process these simultaneous computations in parallel. As parallel processing would be an important factor in achieving an efficiency product it was decided that a range of programming paradigms should be researched in order to determine the most effective methods in achieving performance increases in this manner. The project proposal concluded as the production of an ant colony simulation with an emphasis on designing code, while exploring multiple paradigms, to produce a product that scaled as the problem size increased, through the use of parallel algorithms.

%------Why it entails what it should entail
\paragraph{Relevance to Degree} During the early stages of the project much reading was done on topics related to Artificial Intelligence, and the research in this area inspired the choice of project. This research also aided learning later in the course when CI342 - Advanced Artificial Intelligence was a required module. The project also covers the design of algorithms thus many of the topics covered in the CI312 - Computer Graphics Algorithms module provided valuable information in regard to measuring the time complexity of an algorithm. This module's assignment also encouraged the implementation of algorithms based on information presented in academic papers. The experience of successfully programming algorithms from this approach provided confidence when researching new techniques. Research techniques applied throughout the project were also developed in the module CI339 - Emerging Game Technologies, which involved an investigation and the production of an academic paper based on an emerging trend. CI346 - Programming Languages, Concurrency and Client-Server Computing was a module which not only greatly aided the study and development within this project but knowledge acquired throughout this process has also been of great assistance in the CI346 module. For instance, implementing a solution to a problem in the statically typed programming language ADA was far easier to realise than it would have been without the knowledge and experience gained within the project.

%------Major Aspects
\paragraph{Major Aspects}As identified in the preparation of the proposal there are two major aspects to this project, a simulation and multi-core programming. Simulations are sometimes seen as a genre of game; a life simulation game could revolve around a particular character and its relationships within the simulation, or it could be a simulation of an ecosystem \cite{Spore2009}. Biological simulations often allow the player to experiment with genetics, survival or ecosystems; this can often be for the purposes of education. Unlike other genres of games, simulation games do not usually have set goals or end conditions that allow a player to win the game. Rather, they focus on the experience of control whether it be the lives of people, when micromanaging a family (the most notable example of this is Will Wright's The Sims), to the overseeing the rise of a civilization or success of a business. Outside of recreational and educational games, biological simulations are utilized by researchers to test theories, which in practise would be impossible to carry out because of technological or financial constraints. It also gives the opportunity to show research findings visually. These are often large scale simulations which demand large amounts of processing power due to the nature of the problems. For example, neurological simulation in the Blue Brain Project\cite{Graham-Rowe2005} or weather prediction \cite{Michalakes2008}.

The second aspect of the project is multi-core programming - Processor clockspeeds aren't increasing at the rate they used to, with the focus of hardware manufacturers now producing processors consisting of multiple cores\cite{Sutter2005}. This shift in focus still keeps Moore's law intact, but order to harness the full potential of these multi-core processors, it is necessary to run computations on all of the processors' available cores. While it possible to describe the splitting up of work between a processor's cores as a trivial process, in practice it opens up several new issues which a programmer must address; into how many pieces should the computation be split, will the smaller computations need to communicate their results to other computations, is it necessary for some computations to finish executing before others begin? These are all questions which a programmer faces when designing an application to make use of a multi-core processor. There are two words which often occur in discussions about multi-core programming, Parallelism and Concurrency. It is important to clarify these terms when addressing the questions multi-core programming poses and further research will attempt to achieve this clarity.

%------Summary of the Project CHECK THE CHAPTER NUMBER TODO Something more?
\paragraph{Summary}
Although qualitative objectives were not achieved in their entirety the project has proven an academic success, as the knowledge gained and the potential for further work makes it a good candidate for continued research. The experiences gained over the course of the project now allows for the reasoning of new programming problems from a different perspective. Also the produced artefact, although incomplete has the potential to yield interesting results in the field of parallel processing. A detailed chronological and evaluative description of the implementation process can be found in Chapter 5 of this document. The following is a brief summary of the documents structure.

%------Future Chapters DOUBLE CHECK NUMBERS AND DESCRIPTIONS ARE STILL GOOD WHEN FINISHED TODO
In Chapter 2 important concepts in parallelism and information from relevant publications on the topic is gathered. Studying them provided insights about the inherent problems in the development of large scale simulations and their possible solutions.

Chapter 3 is focused on specifying the goals of this project and on analysis of the algorithms that are going to be used as a basis of the application.

Chapter 4 details the evolution of the design for the final product explaining and justifying the architectural and design choices that were made.

Chapter 5 describes the progress made up to the current stage of the development process. Specific problems that appeared during implementation and testing and suggestions for their resolution are examined.

Chapter 6 shows and discusses the findings made and the level up to which the project goals were achieved. It also mentions possible improvements for the simulation and suggests new areas of investigation prompted by the lessons learned and the inevitable evolution of hardware.

Chapter 7 gives a brief summary of the points made in the previous chapters.

%% End of Intro Content
\end{document}

