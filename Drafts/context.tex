\documentclass[main.tex]{subfiles} 
\begin{document}

Context and Background Research

\section{Research overview} %% CHECK
In this project the problem of simulating the independant movements of a group of ants is tackled, instead of taking a short-cut approach to the problem and allowing every ant to see the whole 'world' the ants will have a realistic limited field of sense. With each ant making decions based on their their own tasks, independantly; while sharing the common goals of the colony. This will provide several problems which the project will attempt to address. Over the course of the project there will be a particular focus how large the simulation can be while still running efficiently. This will take into account the amount of ants working individually yet simultaneously, while trying to keep the nature of their interactions as accurate as possible, as well as the size of the simulation world the ants are contained in.

However it is important to mention here that the realisitic behaviour of an ant and ant colony is not the aim of this project. Rather, the aim is to produce a simulation which exploits the parallelism offered by the ant colony problem. Therefore, a description of the behaviour of ants and the ant colony will be given below which should be agreeable with the average non-entomologist. After, there will be an analysis and discussion of various approaches to tackling the problem that has been posed, not only abstractly but programatically. Firstly looking at various programming paradigms before looking at different simulations. Then research will continue further on methods mentioned in this disscussion that seem most likely to provide a solution. 

\section{Ant Colony Generalizations} %% TODO Write generalization



\section{Problem Area}
When the concept of this simulation is broken down it is evident that a crutial element of the system to be produced is the ant. As mentioned before, the amount of ants in a colony can grow to a very large number, within this simulation however in order to produce a working final product an incremental approach will be taken to solve the problem loooking at first representing small amounts of ants, then looking at getting collision detection working between them before increasing the complexity of an ant and increasing the amount of ants in the system. The project will be aiming to take an approach that scales well, so that when more ants are added to the simulation the performance doesn't drop to a point where it is no longer accurate. This is important to note as when we look for a programming approach in the research stage of the paper it would be sensible choose one that inherently produces solutions capable of scaling well so that performance doesn't dip as the demand on the system is increased.
% TODO This doesn't blend well the Lines before too!
With this in mind a notable observation that can be made of the problem is that it is rich in concurrency. Concurrency is a property attributted to a system which performs more than one ``possibly unrelated tasks at the same time.'' \cite{OSullivan2008} Holding this definition against my problem it is clear that this problem is a concurrent one when viewing ants as tasks.

\chapter{Research}
\section {Research Aims}
\paragraph{}Throughout the following section a breakdown of the topics that will be important to the project will be provided. Questions will then be derived from these topics in order to produce a set of aims that the research will aim to address. When tackling a concurrent problem it would profitable to research what are the most popular and promising programming paradigms used to approach parallel and concurrent problems, taking this further to look at why these paradigms are more efficient at dealing with such problems. Getting more specific, the focus will then be turned to what programming languages are used for this type of problem, again looking at why certain languages are perferred. Other important aspects of the project will also be further researched such as collision detection, the visual representation of the simulation. Most importantly a programming paradigm will be chosen to focus on ``the choice of programming paradigm can significantly influence the way one thinks about problems and expresses solutions'' to problems. \cite{Curriculum2008}

\paragraph{}To summarise the research aims of this project are as follows:
\begin{itemize}
	\item Identify a programming paradigm that is suitable for the problem.
	\item Compare languages that could be used in the implementation of the project.
	\item Review algorithms and Data Structures (represented in languages which may be used throughout this project) that tackle concurrency and other problems found in my project.
	\item Analyze other large scale simulations.
	\item Look at approaches to large scale collision detection.
	\item Consider possible ways to represent the information in the simulation visually.
\end{itemize}



%Looking for a language
\section{Programming Paradigms}
\paragraph{}``Over the last decades, several programming paradigms emerged and profiled. The most important ones are: imperative, object-oriented, functional, and logic paradigm.''\cite{Vujosevic2008} The next few paragraphs will look breifly at these four paradigms and go on to mention a few others which might be of interest considering the problem.

\subsection{The Imperative Paradigm}
The imperative programming paradigm is based on the Von Neumann architecture of computers, introduced in 1940’s. \cite{Vujosevic2008} This means the programming paradigm similar to the Von Neumann machine operate by performing one operation at a time, on a certain pieces of data retrieved from memory, in sequential order. According to Backus \cite{Backus1978} the man who cointed the term ``The von Neumann bottleneck'', ``there are several problems created by the word-at-a-time von Neumann style of programming, with its primitive use of loops, subscripts, and branching flow of control.'' The essence of the Von Neumann architecture is the concept of a modifiable storage. Variables and assignments are the programming language representation of this modifiable storage. The storage is then is manipulated by the program in just as the variables and assignment statements with in the program dictate. Imperative programming languages provide a variety of commands to provide structure to code and to manipulate the store.\cite{Aaby1996}

%% TODO TIDY
Each imperative programming language defines a particular view of hardware.

These views are so distinct that it is common to speak of a Pascal machine, C machine or a Java machine. A compiler implements the virtual machine defined by the programming language in the language supported by the actual hardware and operating system.


As far as we were aware, we simply made up the language as we went along. We did not regard language design as a difficult problem, merely a simple prelude to the real problem: designing a compiler which could produce efficient programs [In R. L. Wexelblat, History of Programming Languages, Academic Press, 
1981, page 30.]

\subsection{The Object Orientated Paradigm}
The Object orientated paradigm focuses on, as its name suggests, elements of a program which it calls objects. Objects are groups of similar data and functionality related to that data held by the object. The data and functions which are grouped within an object can then have their access restricted from functionas and data contained in other objects. This process is commonly called encapsulation and is common in the Object orientated paradigm. Encapsulation allows for a more formal structuring of a program than imperative languages however, most object orientated languages still have mutable state within each object. The way in which this mutable state is manipulated is often very similar to that of imperative languages.\cite{Aaby1996} 

As well as the concept of encapsulation Object Orientated languages introduce other concepts such as Abstraction and Inheritance which in order to implement, encourage the programmer to look for areas where the code they are writing could be reused. \cite{DavidJ.Barnes2008} The modular structure of code which results from the correct use of Object orientated programming techniques means that these reusable modules can be tested seperately too allowing the ability for tests to be focused on modules can result in a greater amount of more specific tests which are more likely to catch errors in code.

%% TODO INCLUDE
``It is our basic belief that extreme caution is warranted when designing and building multi-threaded applications ... use of threads can be very deceptive ... in almost all cases they make debugging, testing, and maintenance vastly more difficult and sometimes impossible.  Neither the training, experience, or actual practices of most programmers, nor the tools we have to help us, are designed to cope with the non-determinism ... this is particularly true in Java ... we urge you to think twice about using threads in cases where they are not absolutely necessary ...'' %http://java.sun.com/products/jfc/tsc/articles/threads/threads1.html Accessed 7th March 201  

\subsection{The Functional Paradigm} %%TODO TIDY!
Object orientated programming has been described as ``the antithesis of functiontional programming'' \cite{Taivalsaari1993} From the two programming paradigms above it can be deduced that a good programming language is modular. But John Hughes claims that it is not enough for a language to just support modularity, it needs to go a step further and make modular programming easy. To do this a programming language needs to provide flexible functionality in order to bring modules together. In an an article on Functional Programming Hughes writes ``Functional programming languages provide two new kinds of glue - higher-order functions and lazy evaluation.''\cite{Hughes1984} 


%
Functions play an important role in functional programming languages. Functions are considered to be values just like integers or strings. A function can return another function, it can take a function as a parameter, and it can even be constructed by composing functions. This offers a stronger "glue" to combine the modules of your program. A function that evaluates some expression can take part of the computation as an argument for instance, thus making the functio modular to a futher extent.
You could also have a function construct another function. For instance, you could define a function "differentiate" that will differentiate a given function numerically. So if you then have a function "f" you could define "f' = differentiate f", and use it like you would normally in a mathematical context. These types of functions are called higher order functions.
%
The title 'higher order` is the title given to functions which follow the lambda calculus representation of functions\cite{Duame2002}. In lambda calculus allows functions to be partially applied
below is an example of a function which squares the value passed as a paremeter in the purely functional language haskell.

Lazy evaluation is the result of not evaluating a functions result until its result is needed. This means that arguments are not evaluated before being passed into a function, but only when their values are actually used, allowing functions to make use of infinite loops and ignore undefined values. 

Hightlighting the differences between the Functional paradigm
http://msdn.microsoft.com/en-us/library/bb669144.aspx
%% discuss TODO purity

\subsection{The Logical Paradigm} %%GIVITSUMMO'
Logic programming is characterized by its use of relations and inference.\cite{Aaby1996} The logic programming paradigm influences have led to the the creation of deductive databases which enhance relational databases by providing deduction capabilities. The logic programming paradigm can be summarised with its three main features. Firstly computation occurs over a domain of terms dfined in a ``universal alphabet''. Secondly, values are assigned to variables through atomatic substitutions called ``most general unifiers'', in some situations these assigned values may themselves be variables an are called logical variables \cite{Apt2001}. Finally the control of a program in the logical paradigm comes from backtracking alone. Here lies the reasons for both the strengths and weaknesses of the logical paradigm. Its strengths come in the form of great simplicity and conciseness, while its weakness lies in having only ``one control mechanism'' and ``a single datatype \cite{Apt2001}.

\section{Development Language}

%% TODO
Based on the research the benifits of each programming paradigm it appears that the functional programming paradigm has more to offer a developer looking to produce a solution to a parallel problem.

Functional work differntly from imperrative Rather than performing actions in a sequence, they evaluate expressions

\subsection{Erlang}

Joe Armstrong\cite{Armstrong2009} has emphasised, when talking about the language that he created, that robustness is a key aim of Erlang. The language was developed by the telecoms company Ericsson during a pursuit for a better way of approaching things. Safety and error management are very important environments such as telecoms as downtime is something to be avoided at all costs. Such is th focus on error handling in Erlang that there are three kinds of exceptions, errors, throws and exits; these all have different uses\cite{Trottier-Hebert2011}. Proper use of Erlang's extensive error handling systems can be used to program robust applications. Another intersting and novel feature of Erlang is its support for continuous operation. The language has primitives which allow code to be replaced in a running system, allowing different versions of the code to execute at the same time \cite{Armstrong1995}. This is extremely benficial for systems which ideally shouldn't stop such as, telephone exchanges or air traffic control systems. %In these cases the system cannot be halted to make changes in the software.

Memory is allocated automatically when required, and deallocated when no longer used. So typical programming errors caused by bad memory management will not occur. This is becase Erlang is a memory managed language with a real-time garbage collector \cite{Armstrong1995}. Erlang was initially implemented in Prolog which may explain its declarative syntax and its use of pattern matching\cite{Armstrong2009}. The language also has a dynamic type system meaning the majority of its type checking is performed at run-time as opposed to during compile-time.

This means errors the programmer makes concerning variable types may occur at runtime, possibly quite distant from the place where the programming mistake was made. Although dynamic typing  may make it easier to get code compiling in the first instance it may also make bugs difficult to locate later on in the development process. Erlang has a process-based model of concurrency using asynchronous message passing. The concurrency mechanisms in Erlang are processes require little memory (lightweight), and creating and deleting processes and message passing require little computational effort \cite{Armstrong1995}. These benifits are due in part to the fact that Erlang is intended for programming soft real-time systems where response times are required to be within miliseconds.

With its strengths in reliability, error handlinhg and light weight concurrencty Erlang naturally excells at distributive programming. Erlang has no shared memory,\cite{Armstrong{1995} so all interaction between processes is done through asynchronous message passing. Due to its use of the Actor Model which is discussed in further research, distributed systems can easily be built in Erlang. As a result of the Erlang treating the other Erlang processes as ``black boxes'' with regards to passing messages to them, the language can integrate; easily calling or make use of programs written in other programming languages. These programs can even be interfaced to the system in such a way that they even appear to the programmer as though they were written in Erlang\cite{Armstrong{1995}.

\subsection{Haskell}
%%

%The level of abstraction
There are two areas that are fundamental to programming a computer - resource management and sequencing. Resource management (allocating registers and memory) has been the target of vast abstraction, most new languages (imperative as well as functional) have implemented garbage collection to remove resource management from the problem, and lets the programmer focus on the algorithm instead of the book-keeping task of allocating memory. Sequencing has also undergone some abstraction, although not nearly to the same extent. Imperative languages have done so by introducing new keywords and standard libraries. For example, most imperative languages have special syntax for constructing several slightly different loops, you no longer have to do all the tasks of managing these loops yourself. But imperative languages are based upon the notion of sequencing - they can never escape it completely.

The only way to raise the level of abstraction in the sequencing area for an imperative language is to introduce more keywords or standard functions, thus cluttering up the language. This close relationship between imperative languages and the task of sequencing commands for the processor to execute means that imperative languages can never rise above the task of sequencing, and as such can never reach the same level of abstraction that functional programming languages can.

***In Haskell, the sequencing task is removed. You only care what the program is to compute not how or when it is computed. This makes Haskell a more flexible and easy to use language.

%Functions and side-effects in functional languages

Here is a short Haskell example of a function numOf that counts the number of elements in a list that satisfy a certain property.

numOf p xs = length (filter p xs)
We will discuss Haskell syntax later, but what this line says is just "To get the result, filter the list xs by the test p and compute the length of the result". Now p is a function that takes an element and returns True or False determining whether the element passes or fails the test. So numOf is a higher order function, some of the functionality is passed to it as an argument. Notice that filter is also a higher order function, it takes the "test function" as an argument. Let's play with this function and define some more specialized functions from it.

numOfEven xs = numOf even xs
Here we define the function numOfEven which counts the number of even elements in a list. Note that we do not need to explicitly declare xs as a parameter. We could just as well write numOfEven = numOf even. A very clear definition indeed. But we'll explicitly type out the parameters for now.

Let's define a function which counts the number of elements that are greater or equal to 5 :

numOfGE5 xs = numOf (>=5) xs
Here the test function is just ">=5" which is passed to numOf to give us the functionality we need.

Hopefully you should now see that the modularity of functional programming allows us to define a generic functions where some of the functionality is passed as an argument, which we can later use to define shorthands for any specialized functions. This small example is somewhat trivial, it wouldn't be too hard to re-write the function definition for all the functions above, but for more complex functions this comes in handy. You can, for instance, write only one function for traversing an auto-balancing binary tree and have it take some of the functionality as a parameter (for instance the comparison function). This would allow you to traverse the tree for any data type by simply providing the relevant comparison function for your needs. Thus you can expend some effort in making sure the general function is correct, and then all the specialized functions will also be correct. Not to mention you wouldn't have to copy and paste code all over your project. This concept is possible in some imperative languages as well. In some object oriented languages you often have to provide a "Comparator object" for trees and other standard data structures. The difference is that the Haskell way is a lot more intuitive and elegant (creating a separate type just for comparing two other types and then passing an object of this type is hardly an elegant way of doing it), so it's more likely to be used frequently (and not just in the standard libraries).

A central concept in functional languages is that the result of a function is determined by its input, and only by its input. There are no side-effects! This extends to variables as well - variables in Haskell do not vary. This may sound strange if you're used to imperative programming (where most of the code consists of changing the "contents" of a variable), but it's really quite natural. A variable in Haskell is a name that is bound to some value, rather than an abstraction of some low-level concept of a memory cell like in imperative languages. When variables are thought of as short-hands for values (just like they are in mathematics), it makes perfect sense that variable updates are not allowed. You wouldn't expect "4 = 5" to be a valid assignment in any language, so it's really quite strange that "x = 4; x = 5" is. This is often hard to grasp for programmers who are very used to imperative languages, but it isn't as strange as it first seems. So when you start thinking things like "This is too weird, I'm going back to C++!", try to force yourself to continue learning Haskell - you'll be glad you did.

Removing side-effects from the equation allows expressions to be evaluated in any order. A function will always return the same result if passed the same input - no exceptions. This determinism removes a whole class of bugs found in imperative programs. In fact, you could even argue that most bugs in large systems can be traced back to side-effects - if not directly caused by them, then caused by a flawed design that relies on side-effects. This means that functional programs tend to have far fewer bugs than imperative ones.

1.3 Conclusion
Because functional languages are more intuitive and offer more and easier ways to get the job done, functional programs tend to be shorter (usually between 2 to 10 times shorter). The semantics are most often a lot closer to the problem than an imperative version, which makes it easier to verify that a function is correct. Furthermore Haskell doesn't allow side-effects, which leads to fewer bugs. Thus Haskell programs are easier to write, more robust, and easier to maintain.

2 What can Haskell offer the programmer?
Haskell is a modern general purpose language developed to incorporate the collective wisdom of the functional programming community into one elegant, powerful and general language.

2.1 Purity
See FP***************

Unlike some other functional programming languages Haskell is pure. It doesn't allow any side-effects. This is probably the most important feature of Haskell. We've already briefly discussed the benefits of pure, side-effect free, programming - and there's not much more we can say about that. You'll need to experience it yourself.

2.2 Laziness
Another feature of Haskell is that it is lazy (technically speaking, it's "non-strict"). This means that nothing is evaluated until it has to be evaluated. You could, for instance, define an infinite list of primes without ending up in infinite recursion. Only the elements of this list that are actually used will be computed. This allows for some very elegant solutions to many problems. A typical pattern of solving a problem would be to define a list of all possible solutions and then filtering away the illegal ones. The remaining list will then only contain legal solutions. Lazy evaluation makes this operation very clean. If you only need one solution you can simply extract the first element of the resulting list - lazy evaluation will make sure that nothing is needlessly computed.

2.3 Strong typing
Furthermore Haskell is strongly typed, this means just what it sounds like. It's impossible to inadvertently convert a Double to an Int, or follow a null pointer. This also leads to fewer bugs. It might be a pain in the neck in the rare cases where you need to convert an Int to a Double explicitly before performing some operation, but in practice this doesn't happen often enough to become a nuisance. In fact, forcing each conversion to be explicit often helps to highlight problem code. In other languages where these conversions are invisible, problems often arise when the compiler treats a double like an integer or, even worse, an integer like a pointer.

Unlike other strongly typed languages types in Haskell are automatically inferred. This means that you very rarely have to declare the types of your functions, except as a means of code documentation. Haskell will look at how you use the variables and figure out from there what type the variable should be - then it will all be type-checked to ensure there are no type-mismatches. Python has the notion of "duck typing", meaning "If it walks and talks like a duck, it's a duck!". You could argue that Haskell has a much better form of duck typing. If a value walks and talks like a duck, then it will be considered a duck through type inference, but unlike Python the compiler will also catch errors if later on it tries to bray like a donkey! So you get the benefits of strong typing (bugs are caught at compile-time, rather than run-time) without the hassle that comes with it in other languages. Furthermore Haskell will always infer the most general type on a variable. So if you write, say, a sorting function without a type declaration, Haskell will make sure the function will work for all values that can be sorted.

Compare how you would do this in certain object oriented languages. To gain polymorphism you would have to use some base class, and then declare your variables as instances of subclasses to this base class. It all amounts to tons of extra work and ridiculously complex declarations just to proclaim the existence of a variable. Furthermore you would have to perform tons of type conversions via explicit casts - definitely not a particularly elegant solution. If you want to write a polymorphic function in these object oriented languages you would probably declare the parameters as an object of a global base class (like "Object" in Java), which essentially allows the programmer to send anything into the function, even objects which can't logically be passed to the function. The end result is that most functions you write in these languages are not general, they only work on a single data type. You're also moving the error checking from compile-time to run-time. In large systems where some of the functionality is rarely used, these bugs might never be caught until they cause a fatal crash at the worst possible time.

Haskell provides an elegant, concise and safe way to write your programs. Programs will not crash unexpectedly, nor produce strangely garbled output.

2.4 Elegance
Another property of Haskell that is very important to the programmer, even though it doesn't mean as much in terms of stability or performance, is the elegance of Haskell. To put it simply: stuff just works like you'd expect it to.

To highlight the elegance of Haskell we shall now take a look at a small example. We choose QuickSort-inspired filtering sort because it's a simple algorithm that is actually useful. We will look at two versions - one written in C++, an imperative language, and one written in Haskell. Both versions use only the functionality available to the programmer without importing any extra modules (otherwise we could just call "sort" in each language's standard library and be done with it!). Thus, we use the standard sequence primitives of each language (a "list" in Haskell and an "array" in C++). Both versions must also be polymorphic (which is done "automatically" in Haskell, and with templates in C++). Both versions must use the same recursive algorithm.

Please note that this is not intended as a definite comparison between the two languages. It's intended to show the elegance of Haskell, the C++ version is only included for comparison (and would be coded quite differently if you used the standard QuickSort algorithm or the Standard Template Libraries (STL), for example).

template <typename T>
void qsort (T *result, T *list, int n)
{
    if (n == 0) return;
    T *smallerList, *largerList;
    smallerList = new T[n];
    largerList = new T[n];      
    T pivot = list[0];
    int numSmaller=0, numLarger=0;      
    for (int i = 1; i < n; i++)
        if (list[i] < pivot)
            smallerList[numSmaller++] = list[i];
        else 
            largerList[numLarger++] = list[i];
    
    qsort(smallerList,smallerList,numSmaller); 
    qsort(largerList,largerList,numLarger);
    
    int pos = 0;        
    for ( int i = 0; i < numSmaller; i++)
        result[pos++] = smallerList[i];
    
    result[pos++] = pivot;
    
    for ( int i = 0; i < numLarger; i++)
        result[pos++] = largerList[i];
    
    delete [] smallerList;
    delete [] largerList;
};
We will not explain this code further, just note how complex and difficult it is to understand at a glance, largely due to the programmer having to deal with low-level details which have nothing to do with the task at hand. Now, let's take a look at a Haskell version of FilterSort, which might look a something like this.

qsort :: (Ord a) => [a] -> [a]
 qsort []     = []
 qsort (x:xs) = qsort less ++ [x] ++ qsort more
     where less = filter (<x)  xs
           more = filter (>=x) xs
(This implementation has very poor runtime and space complexity, but that can be improved, at the expense of some of the elegance.)

Let's dissect this code in detail, since it uses quite a lot of Haskell syntax that you might not be familiar with.

The first line is a type signature. It declares "qsort" to be function that takes a list "[a]" as input and returns ("->") another list "[a]". "a" is a type variable (vaguely similar to a C++ template declaration), and "(Ord a)" is a constraint that means that only types that have an ordering are allowed. This function is a generic ("template") function, that can sort any list of pairwise-comparable objects. The phrase "(Ord a) => [a] -> [a]" means "if the type 'a' is ordered, than a list of 'a' can be passed in, and another list of 'a' will come out."

The function is called qsort and takes a list as a parameter. We define a function in Haskell like so: funcname a b c = expr, where funcname is the name of the function, a, b, and, c are the parameters and expr is the expression to be evaluated (most often using the parameters). Functions are called by simply putting the function name first and then the parameter(s). Haskell doesn't use parenthesis for function application. Functions simply bind more tightly than anything else, so "f 5 * 2", for instance, would apply f to 5 and then multiply by 2, if we wanted the multiplication to occur before the function application then we would use parenthesis like so "f (5*2)".

Let's get back to FilterSort. First we see that we have two definitions of the functions. This is called pattern matching and we can briefly say that it will test the argument passed to the function top-to-bottom and use the first one that matches. The first definition matches against [] which in Haskell is the empty list (a list of 1,2 and 3 is [1,2,3] so it makes sense that an empty list is just two brackets). So when we try to sort an empty list, the result will be an empty list. Sounds reasonable enough, doesn't it? The second definition pattern matches against a list with at least one element. It does this by using (x:xs) for its argument. The "cons" operator is (:) and it simply puts an element in front of a list, so that 0 : [1,2,3] returns [0,1,2,3]. Pattern matching against (x:xs) is a match against the list with the head x and the tail xs (which may or may not be the empty list). In other words, (x:xs) is a list of at least one element. So since we will need to use the head of the list later, we can actually extract this very elegantly by using pattern matching. You can think of it as naming the contents of the list. This can be done on any data construct, not just a list. It is possible to pattern match against an arbitrary variable name and then use the head function on that to retrieve the head of the list. Now if we have a non empty list, the sorted list is produced by sorting all elements that are smaller than x and putting that in front of x, then we sort all elements larger than x and put those at the end. We do this by using the list concatenation operator ++. Notice that x is not a list so the ++ operator won't work on it alone, which is why we make it a singleton-list by putting it inside brackets. So the function reads "To sort the list, sandwich the head between the sorted list of all elements smaller than the head, and the sorted list of all elements larger than the head". Which could very well be the original algorithm description. This is very common in Haskell. A function definition usually resembles the informal description of the function very closely. This is why I say that Haskell has a smaller semantic gap than other languages.

But wait, we're not done yet! How is the list less and more computed? Well, remember that we don't care about sequencing in Haskell, so we've defined them below the function using the where notation (which allows any definitions to use the parameters of the function to which they belong). We use the standard prelude function filter, I won't elaborate too much on this now, but the line less = filter (<x) xs will use filter (<x) xs to filter the list xs. You can see that we actually pass the function which will be used to filter the list to filter, an example of higher order functions. The function (<x) should be read out "the function 'less than x'" and will return True if an element passed to it is less than x (notice how easy it was to construct a function on the fly, we put the expression "<x", "less than x", in parenthesis and sent it off to the function - functions really are just another value!). All elements that pass the test are output from the filter function and put inside less. In a same way (>=x) is used to filter the list for all elements larger than or equal to x.

Now that you've had the syntax explained to you, read the function definition again. Notice how little time it takes to get an understanding about what the function does. The function definitions in Haskell explain what it computes, not how.

If you've already forgotten the syntax outlined above, don't worry! We'll go through it more thoroughly and at a slower pace in the tutorials. The important thing to get from this code example is that Haskell code is elegant and intuitive.

2.5 Haskell and bugs
We have several times stated that various features of Haskell help fight the occurrence of bugs. Let's recap these.

Haskell programs have fewer bugs because Haskell is:

Pure. There are no side effects.
Strongly typed. There can be no dubious use of types. And No Core Dumps!
Concise. Programs are shorter which make it easier to look at a function and "take it all in" at once, convincing yourself that it's correct.
High level. Haskell programs most often reads out almost exactly like the algorithm description. Which makes it easier to verify that the function does what the algorithm states. By coding at a higher level of abstraction, leaving the details to the compiler, there is less room for bugs to sneak in.
Memory managed. There's no worrying about dangling pointers, the Garbage Collector takes care of all that. The programmer can worry about implementing the algorithm, not book-keeping of the memory.
Modular. Haskell offers stronger and more "glue" to compose your program from already developed modules. Thus Haskell programs can be more modular. Often used modular functions can thus be proven correct by induction. Combining two functions that are proven to be correct, will also give the correct result (assuming the combination is correct).
Furthermore most people agree that you just think differently when solving problems in a functional language. You subdivide the problem into smaller and smaller functions and then you write these small (and "almost-guaranteed-to-be-correct") functions, which are composed in various ways to the final result. There just isn't any room for bugs!


3 Haskell vs OOP
The great benefit of Object Oriented Programming is rarely that you group your data with the functions that act upon it together into an object - it's that it allows for great data encapsulation (separating the interface from implementation) and polymorphism (letting a whole set of data types behave the same way). However:

Data encapsulation and polymorphism are not exclusive to OOP!

Haskell has tools for abstracting data. We can't really get into it without first going through the module system and how abstract data types (ADT) work in Haskell, something which is well beyond the scope of this essay. We will therefore settle for a short description of how ADTs and polymorphism works in Haskell.

Data encapsulation is done in Haskell by declaring each data type in a separate module, from this module you only export the interface. Internally there might be a host of functions that touch the actual data, but the interface is all that's visible from outside of the module. Note that the data type and the functions that act upon the data type are not grouped into an "object", but they are (typically) grouped into the same module, so you can choose to only export certain functions (and not the constructors for the data type) thus making these functions the only way to manipulate the data type - "hiding" the implementation from the interface.

Polymorphism is done by using something called type classes. Now, if you come from a C++ or Java background you might associate classes with something resembling a template for how to construct an object, but that's not what they mean in Haskell. A type class in Haskell is really just what it sounds like. It's a set of rules for determining whether a type is an instance of that class. So Haskell separates the class instantiation and the construction of the data type. You might declare a type "Porsche", to be an instance of the "Car" type class, say. All functions that can be applied onto any other member of the Car type class can then be applied to a Porsche as well. A class that's included with Haskell is the Show type class, for which a type can be instantiated by providing a show function, which converts the data type to a String. Consequently almost all types in Haskell can be printed onto the screen by applying show on them to convert them to a String, and then using the relevant IO action (more on IO in the tutorials). Note how similar this is to the the object notion in OOP when it comes to the polymorphism aspect. The Haskell system is a more intuitive system for handling polymorphism. You won't have to worry about inheriting in the correct hierarchical order or to make sure that the inheritance is even sensible. A class is just a class, and types that are instances of this class really doesn't have to share some parent-child inheritance relationship. If your data type fulfills the requirements of a class, then it can be instantiated in that class. Simple, isn't it? Remember the QuickSort example? Remember that I said it was polymorphic? The secret behind the polymorphism in qsort is that it is defined to work on any type in the Ord type class (for "Ordered"). Ord has a set of functions defined, among them is "<" and ">=" which are sufficient for our needs because we only need to know whether an element is smaller than x or not. So if we were to define the Ord functions for our Porsche type (it's sufficient to implement, say, <= and ==, Haskell will figure out the rest from those) in an instantiation of the Ord type class, we could then use qsort to sort lists of Porsches (even though sorting Porsches might not make sense). Note that we never say anything about which classes the elements of the list must be defined for, Haskell will infer this automatically from just looking at which functions we have used (in the qsort example, only "<" and ">=" are relevant).

So to summarize: Haskell does include mechanisms for data encapsulation that match or surpass those of OOP languages. The only thing Haskell does not provide is a way to group functions and data together into a single "object" (aside from creating a data type which includes a function - remember, functions are data!). This is, however, a very minor problem. To apply a function to an object you would write "func obj a b c" instead of something like "obj.func a b c".


4 Modularity
A central concept in computing is modularity. A popular analogy is this: say you wanted to construct a wooden chair. If you construct the parts of it separately, and then glue them together, the task is solved easily. But if you were to carve the whole thing out of a solid piece of wood, it would prove to be quite a bit harder. John Hughes had this to say on the topic in his paper Why Functional Programming Matters

"Languages which aim to improve productivity must support modular programming well. But new scope rules and mechanisms for separate compilation are not enough - modularity means more than modules. Our ability to decompose a problem into parts depends directly on our ability to glue solutions together. To assist modular programming, a language must provide good glue.

Functional programming languages provide two new kinds of glue - higher-order functions and lazy evaluation."


5 The speed of Haskell
Let me first state clearly that the following only applies to the general case in which speed isn't absolutely critical, where you can accept a few percent longer execution time for the benefit of reducing development time greatly. There are cases when speed is the primary concern, and then the following section will not apply to the same extent.

Now, some C++ programmers might claim that the C++ version of QuickSort above is probably a bit faster than the Haskell version. And this might be true. For most applications, though, the difference in speed is so small that it's utterly irrelevant. For instance, take a look at the Computer Language Benchmarks Game, where Haskell compares favorably to most of the so called "fast" languages. Now, these benchmarks don't prove all that much about real-world performance, but they do show that Haskell isn't as slow as some people think. At the time of writing it's in 4th position, only slightly behind C and C++.

Almost all programs in use today have a fairly even spread of processing time among its functions. The most notable exceptions are applications like MPEG encoders, and artificial benchmarks, which spend a large portion of their execution time within a small portion of the code. If you really need speed at all costs, consider using C instead of Haskell.

There's an old rule in computer programming called the "80/20 rule". It states that 80\% of the time is spent in 20\% of the code. The consequence of this is that any given function in your system will likely be of minimal importance when it comes to optimizations for speed. There may be only a handful of functions important enough to optimize. These important functions could be written in C (using the excellent foreign function interface in Haskell). The role of C could, and probably will, take over the role of assembler programming - you use it for the really time-critical bits of your system, but not for the whole system itself.

We should continue to move to higher levels of abstraction, just like we've done before. We should trade application speed for increased productivity, stability and maintainability. Programmer time is almost always more expensive than CPU time. We aren't writing applications in assembler anymore for the same reason we shouldn't be writing applications in C anymore.

Finally remember that algorithmic optimization can give much better results than code optimization. For theoretical examples when factors such as development times and stability doesn't matter, then sure, C is often faster than Haskell. But in the real world development times do matter, this isn't the case. If you can develop your Haskell application in one tenth the time it would take to develop it in C (from experience, this is not at all uncommon) you will have lots of time to profile and implement new algorithims. So in the "real world" where we don't have infinite amounts of time to program our applications, Haskell programs can often be much faster than C programs.

6 Epilogue
So if Haskell is so great, how come it isn't "mainstream"? Well, one reason is that the operating system is probably written in C or some other imperative language, so if your application mainly interacts with the internals of the OS, you may have an easier time using imperative languages. Another reason for the lack of Haskell, and other functional languages, in mainstream use is that programming languages are rarely thought of as tools (even though they are). To most people their favorite programming language is much more like religion - it just seems unlikely that any other language exists that can get the job done better and faster.

There is an essay by Paul Graham called Beating the Averages describing his experience using Common Lisp, another functional language, for an upstart company. In it he uses an analogy which he calls "The Blub Paradox".

It goes a little something like this: If a programmer's favorite language is Blub, which is positioned somewhere in the middle of the "power spectrum", he can most often only identify languages that are lower down in the spectrum. He can look at COBOL and say "How can anyone get anything done in that language, it doesn't have feature x", x being a feature in Blub.

However, this Blub programmer has a harder time looking the other way in the spectrum. If he examines languages that are higher up in the power spectrum, they will just seem "weird" because the Blub programmer is "thinking in Blub" and can not possibly see the uses for various features of more powerful languages. It goes without saying that this inductively leads to the conclusion that to be able to compare all languages you'll need to position yourself at the top of the power spectrum. It is my belief that functional languages, almost by definition, are closer to the top of the power spectrum than imperative ones. So languages can actually limit a programmers frame of thought. If all you've ever programmed is Blub, you may not see the limitations of Blub - you may only do that by switching to another level which is more powerful.

One of the reasons the mainstream doesn't use Haskell is because people feel that "their" language does "everything they need". And of course it does, because they are thinking in Blub! Languages aren't just technology, it's a way of thinking. And if you're not thinking in Haskell, it is very hard to see the use of Haskell - even if Haskell would allow you to write better applications in a shorter amount of time!

Hopefully this article has helped you break out of the Blub paradox. Even though you may not yet "think in Haskell", it is my hope that you are at least aware of any limitations in your frame of thought imposed by your current "favorite" language, and that you now have more motivation to expand it by learning something new.

\subsection{Scala}

purely OO First, we wanted to be a pure object-oriented language, where every value is an object, every operation is a method call, and every variable is a member of some object. So we didn't want statics, but we needed something to replace them, so we created the construct of singleton objects. But even singleton objects are still global structures. So the challenge was to use them as little as possible, because when you have a global structure you can't change it anymore. You can't instantiate it. It's very hard to test. It's very hard to modify it in any way. [http://www.artima.com/scalazine/articles/goals_of_scala.html]

With the advent of multi-core processors concurrent programming is becoming indispensable. Scala's primary concurrency construct is actors. Actors are basically concurrent processes that communicate by exchanging messages. Actors can also be seen as a form of active objects where invoking a method corresponds to sending a message.
The Scala Actors library provides both asynchronous and synchronous message sends (the latter are implemented by exchanging several asynchronous messages). Moreover, actors may communicate using futures where requests are handled asynchronously, but return a representation (the future) that allows to await the reply. [http://www.scala-lang.org/node/242]

Of the challenges we were facing is we wanted to be both functional and object-oriented. We had very early on the notion that immutable objects would become very, very important. Nowadays everybody talks about immutable objects, because people think they are a key part of the solution to the concurrency problems caused by multi-core computers. Everybody says, no matter what you do, you need to try to have as much of your code using immutable objects as possible. In Scala, we did that very early on. Five or six years ago, we started to think very hard about immutable objects. It actually turns out that a lot of the object-oriented field up to then identified objects with mutability. For them, mutable state and objects were one and the same: mutable state was an essential ingredient of objects. We had to, in essence, ween objects off of that notion, and there were some things we had to do to make that happen.

interoperability with Java (can cause a lapse back to standard OOP programming)

Negatives
Performance differences usually arise from the featf the challenges we were facing is we wanted to be both functional and object-oriented. We had very early on the notion that immutable objects would become very, very important. Nowadays everybody talks about immutable objects, because people think they are a key part of the solution to the concurrency problems caused by multi-core computers. Everybody says, no matter what you do, you need to try to have as much of your code using immutable objects as possible. In Scala, we did that very early on. Five or six years ago, we started to think very hard about immutable objects. It actually turns out that a lot of the object-oriented field up to then identified objects with mutability. For them, mutable state and objects were one and the same: mutable state was an essential ingredient of objects. We had to, in essence, ween objects off of that notion, and there were some things we had to do to make that happen.ures of Scala that are not natively supported by the JVM. Some of these features, such as closures, are likely to be supported soon, but many others never will. Therefore, a lean code at a high level of abstraction written in Scala can be compiled to a large amount of bytecode, degrading runtime performance, as shown in this presentation.


\subsection{Summary choice}

This project will take Haskell as the development language for the solution even though it had not been encountered prior to the initial research of this project, and as such was learnt throughout the duration of the project due to the benifits it offers in regard to the nature of the problem.
% List the benifits with respect to the nature of the problem.
Haskells type system is th richest and most expressive, it is also the strictest on purity while Erlang maintains purity within its processes Haskell forces the programmer to seperate code with side effects from pure code with the use of monads.


\section{Approaches to Parallelism}

\subsection{The Limits of Parralelization}
It has been proved by Gene Amdahl that increasing the amount of processors working on a program only increases speed for so long. Amdahl's law shows that the speedup of a program is limited by the sequential portion of the program. This is clear because sequential processes need to be performed in sequence so by definition they are prevented from benifiting from multiple processors. In 1967 at the AFIPS Spring joint Computer Conference, while talking about the overhead of "data management housekeeping" which can be found in any computer program, Amdahl reportedly said that "The nature of this overhead appears to be sequential so that it is unlikely to be amenable to parallel processing techniques. Overhead alone would then place an upper limit on throughput."\cite{Amdahl1967} This upper limit has been deduced from his conference talk and given rise to the widely known formula below.
% -Amdahl's law latex forumula
The formula calculates the potential speedup that could be brought to a program by adding more processors work on the parallel portion of a problem. The sequential overhead generated in arranging the parallel computation that Amdahl mentions is represented by (1-P) and therefore P representing the parallel section of the program makes up the whole. The forumula then takes the sequential portion and adds it to the parallel portion of the program over N which represents the number of processors this symbolizes the division of the parallel portion of the program between the processors. The upper limit that Amdahl referrs to is clear as you increase the number of processors N, if you had infinity processors working on the parallel part the speed up would simply be 1 over (1-P).
\cite{Amdahl1967}

There are some models around which programmers may base a parallel algorithm around and similarly there are some data structures which .


\subsection{Dynamic Multithreading}

A Program with spawn sync and return statments
A directed asyclic graph can be generated from the runtime of such a program where vertices are each [portion of execution up until a spawn/sync/return statment]

Two performance measures suffice to gauge the theoretical efficiency of multithreaded algorithms. We define the work  of a multithreaded computation to be the total time to execute all the operations in the computation on one processor. We define the critical-path length  of a computation to be the longest time to execute the threads along any path of dependencies in the dag. Consider, for example, the computation in Figure 1. Suppose that every thread can be executed in unit time.

Then, the work of the computation is 17, and the critical-path length is 8. When a multithreaded computation is executed on a given number P of processors, its running time depends on how efficiently the underlying scheduler can execute it. Denote by TP the running time of a given computation on P processors. Then, the work of the computation can be viewed as T1, and the critical-path length can be viewed as T∞. The work and critical-path length can be used to provide lower bounds on the running time on P processors.
We have TP ≥ T1/P , (1)
since in one step, a P-processor computer can do at most P work.
We also have TP ≥ T∞ , (2)

since a P-processor computer can do no more work in one step than an infi nite-processor computer. The speedup of a computation on P processors is the ratio T1/TP , which indicates how many times faster the P-processor execution is than a one-processor execution. If T1/TP = Θ(P), then
we say that the P-processor execution exhibits linear speedup. The maximum possible speedup is T1/T∞, which is also called the parallelism of the computation, because it represents the average amount of work that can be done in parallel for each step along the critical path. We denote the parallelism of a computation by P.


Scheduling
MIT http://www.catonmat.net/blog/mit-introduction-to-algorithms-part-thirteen/

\subsection{The Actor Model}

The Actor Model is based a message passing system between units of computation which are called actors. ``Actors are basically concurrent processes which communicate through asynchronous message passing.'' [http://lampwww.epfl.ch/~odersky/papers/jmlc06.pdf]
Message passing Actors can be creating new actors, decideing what action to take on a message or sending a message to another actor simulatneous to recieving a message. A common xxx in Erlang and Scala

\subsection{Divide and Conquer}

The MapReduce architecture, is an application of the divide and conquer technique. However, useful implementations of the MapReduce architecture will have many other features in place to efficiently "divide", "conquer", and finally "reduce" the problem set. With a large MapReduce deployment (1000's of compute nodes) these steps to partition the work, compute something, and then finally collect all results is non-trivial. Things like load balancing, dead node detection, saving interim state (for long running problems), are hard problems by themselves.

\subsection{Embarassingly Parallel}

Embarassibly Parallel is the term given to problems where the

Other Approaches

Software Transactional Memory

\section{Large Scale Simulations}

an analysis..
What large scale systems are out there
What do they do how do they do it effectively.

Networks
http://www.cs.mcgill.ca/~carl/largescalenetsim.pdf

Molecular dynamics
Millions of atoms http://www.mcs.anl.gov/uploads/cels/papers/scidac11/final/jiang_wei.pdf


Blue brain
Fully implicit parallel simulation of single neuron
experiencing linear speed up
http://bluebrain.epfl.ch/files/content/sites/bluebrain/files/Scientific%20Publications/2008%20-%20Fully%20Implicit%20Parallel%20Simulation%20of%20Single%20Neurons.pdf

Colony Intel
Employs SIMD to also insure instruction level parallelism
http://software.intel.com/en-us/articles/vcsource-samples-colony/




\section {Collision Detection}

Large scale simulations

Different methods

Why each method is effective and disadvantages.

Priori and Posteri / Discrete vs Continuous

Bounding Boxes various types
[diagram]
In addition to the bounding volumes covered here, many other types of volumes
have been suggested as bounding volumes. These include cones [Held97], [Eberly02],
cylinders [Held97], [Eberly00], [Schömer00], spherical shells [Krishnan98], ellipsoids
[Rimon92], [Wang01], [Choi02], [Wang02], [Chien03], and zonotopes [Guibas03].


Partitioning
Bottom-up Construction Strategies
When grouping two objects under a common node, the pair of objects resulting in the smallest bounding volume quite likely corresponds to the pair of objects nearest each other. As such, the merging criterion is often simplified to be the pairing of the query node with its nearest neighbor.

Locating the point (or object) out of a set of points closest to a given query point is known as the nearest neighbor problem. This problem has been well studied, and many different approaches have been suggested. For a small number of objects, a low-overhead brute-force solution is preferred. For larger numbers, solutions based on bucketing schemes or k-d trees are typically the most practical. A k-d tree solution is quite straightforward to implement (see Section 7.3.7 for details). The tree can be built top down in, on average, O(n log n) time, for example by splitting the current set of objects in half at the object median.

Binary Space Partioning

kDimentional Trees

\section {Visual Representation}

OpenGL
2D

\section {Definitions}
%Terms For Definition
%Functional Programming
%Parallelism
%Concurrency

\section{Planning}

\section{Software Development Model}
This project will take an iterative and modular approach to development. This approach can also be seen as ``a way to manage the complexity and risks of large-scale development''\cite{Larman2003}.  It is a useful coding practise to develop small modules which are capable of functioning independantly from the code as these modules can then be reused, in developing other programs. There is also another advantage to be drawn from modular development, each module would be able to be tested sperately which would in turn ensure less bugs when assembling the final product. Another advantage of the iterative incremental approach is that the developer is able to take advantage of what is learned during the development of earlier, increments of the system. As learning comes from both the development and use of the system \cite{Larman2003}.

\section{Important Tasks}
The following is a break down of the project into smaller iterations this can then be used alongside the various project deadlines the project entails, to produce a Gantt chart of how the project's workflow will be carried out.
\begin{itemize}
	\item Representing an Ant
	\item Representing the World
	\item Representing the Pheremone Levels
	\item Individual Ant Intelligence
	\item Collision Detection
	\item Parallelize Algorithms further
	\item Distributing Problem
\end{itemize}

\section{A Schedule of Activities}
The produced Gantt chart attached to this report visualizes the following information. Throughout the following paragraphs a description of the plan for the project will be detailed in the form of a schedule. 

\paragraph{October-November}
Throughout the course of October and November the project will focus mainly on gathering resources  and research. Much of the research will be conducted in the functional language Haskell and small experiments will be made to demonstrate any concepts that have been learnt. Further research will be carried out in areas of the project that will pose potential problems that need to be dealt with later on in the implementation stage of the project. Even at this point in the project the implementation process will begin. As development will be iterative small aims will be set such as representing a single ant and then these aims will be expanded upon.

\paragraph{December-Janurary}
In the months of December and Janurary the code produced in the first section will be expanded upon to produce a working base system. At the end of this this period a simulation of the ant colony should be able to be run. The simulation should include the all the basic functionality of the final system. Ants will be able to move and collide. Each ant will respond to its surroundings by referencing its own basic behaviour tree.

\paragraph{February-March}
In Feburary and March the focus will be on improving the algorithms which have been used within the system. Making them more efficient and looking at structuring more of them to work in parallel. At the end of this period the project's delieverable, the simulation, will be in a state that is presentable with as few bugs as possible. Towards the end of this period if things are on schedule the project's focus will shift from improving the base simulation to adding extentions and making it more usable.

\paragraph{April}
During the final few weeks of the project there will be the fixing of any bugs in the simulation and finishing up any extentions that may be added to the project. During this time the projects focus will be on documentation and producing an informative and evaluative technical report.

\section{Risk Analysis} % Good maybe expand
Throughout this project there are several things that could go wrong, such as  loss of work, hardware failure and the inability to surpass certain bugs which may be encountered throughout the project. To minimize the chances of loss of work source control tools will be utilized in the form of git. This will not only backup the project but keep a track of changes and monitor its evolution. In the event of hardware failiure all the software required for the project is freely available and it would be possible to set up for work either in the University or on a replacement system with minimal effort. There is also a chance that bugs may be encountered which appear impossible to overcome this is a risk that can only be managed through planning and research. Should difficulties be encountered there is a vast range of resources for functional programming online and a Brighton functional programming user group, this would provide a great opportunity to increase my knowledge but should it not be possible to find a solution the only course of action would be to alter the future plans of the project.
%(!TODO) A risk analysis of potential problems.

\section{Social, Ethical and Legal Issues} %TODO
Before embarking on this project it was essential to look into any ethical issues that may arise throughout the course of the project or as a result of using the project's delieverable. 
So far none have no ethical issues have been encountered however ethics  will be kept in mind throughout the course of the project and any issues will be documented in the project log during the project and the technical report at the end of the project.


(A critical appreciation of ethical issueds)
A discussion of the licences that each library is provided under.
Take note of:
Intellectual property.
Research ethics.
Plagiarism.


\end{document}
